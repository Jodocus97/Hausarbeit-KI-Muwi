\documentclass{article}
%%% Packages
\usepackage[ngerman]{babel}
\usepackage{csquotes}
\usepackage[notes, backend=biber]{biblatex-chicago}
\usepackage[a4paper, left=4cm, right=2cm, top=2cm, bottom=4cm]{geometry}
\usepackage[onehalfspacing]{setspace}
\usepackage{float}

%%% Bibliographie
\bibliography{Urheberrecht}

%%% TITEL %%%
\title{Das Plagiat in der Musik unter besonderen Berücksichtigung der Entwicklungen in der Künstlichen Intelligenz}
\author{Patrick Marx}

\begin{document}

\maketitle

\begin{tabular}[b]{l|l}
  Seminar: & Das Urheberrecht in der Musikwissenschaft \\
  \hline
  Kontaktdaten: & Patrick Marx \\
                & Mombacher Straße 60 \\
                & 55122 Mainz \\
                & \\
                & pmarx@students.uni-mainz.de \\
                \hline
  Fachsemester & 2 \\
  Matrikelnummer: & 2745644 \\
  Abgabedatum: & 31.03.2024 \\
\end{tabular}
\thispagestyle{empty}
\newpage

\tableofcontents
\thispagestyle{empty}
\newpage

\setcounter{page}{1}
\section{Einleitung}
Ein Gespenst geht um in der Wissenschaft -- Das Gespenst der Künstlichen Intelligenz\footnote{Frei nach Karl Marx und Friedrich Engels.}. Am 30. November 2022 startete der Chatbot \glqq ChatGPT\grqq~des amerikanischen Unternehmens OpenAI\autocite{schnabelKuenstlicheIntelligenzHausaufgaben2022}. 
Damit startete eine Revolution im Bereich der Künstlichen Intelligenz, denn ChatGPT ist in der Lage aus Eingabeprompts komplexe Texte zu generieren. 
Nicht nur im Bereich der Texterstellung werden KI-Tools immer leistungsstärker, 
auch in der Bildgenerierung lassen sich mit \glqq Midjourney\grqq~oder \glqq DALL-E\grqq~erschreckend realistische Ergebnisse erzielen. 
Auch im Musikbereich wird Künstliche Intelligenz bereits eingesetzt. 
So wurde 2021 mithilfe einer Künstlichen Intelligenz ein Versuch unternommen, Beethovens 10. Sinfonie, die nur in Skizzen überliefert ist, fertigzustellen\autocite{dpaMusikKuenstlicheIntelligenz2021}.

Was verbirgt sich hinter dem Begriff \glqq\textit{Künstliche Intelligenz}\grqq?
Kann eine Maschine tatsächlich lernen? Wie sind computergenerierte Werke vom 
Urheberrecht geschützt? Bedarf es möglicherweise einer Überarbeitung des 
Urheberrechts? Was bedeutet Künstliche Intelligenz für die Künstler:innen? 
All diese Fragen stellen sich, wenn man sich mit Künstlicher Intelligenz in der Kunst beschäftigt. In dieser Arbeit soll versucht werden diese weitestgehend zu beantworten. 
Im ersten Kapitel soll dem Phänomen KI auf den Grund gegangen werden. Dort werden nicht nur die technische Funktionsweise, sondern auch einige philosophische Fragen betrachtet. 
Im zweiten Kapitel wird die rechtliche Perspektive beleuchtet, denn mit dem Einsatz von künstlicher Intelligenz stellen sich zahlreiche
%TODO

\section{Künstliche Inteligenz}
%TODO
Bereits Alan Turing stellte 1950 die Frage \glqq \textit{Can machines think?}\grqq~\autocite[\ppno~433]{turingCOMPUTINGMACHINERYINTELLIGENCE1950} und stellt das sogenannte \glqq \textit{Imitation Game}\grqq~vor. Damit war die Frage nach der Intelligenz von Maschinen bereits gestellt, bevor überhaupt über das Konzept der künstlichen Intelligenz nachgedacht wurde\autocite[\ppno~10]{laemmelKuenstlicheIntelligenzWissensverarbeitung2023}. 
\section{Das Plagiat im Urheberrecht}
%TODO

ASDF
\printbibliography
\end{document}
